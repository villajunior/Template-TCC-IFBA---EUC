%%%%%%%%%%%%%%%%%%%%%%%%%%%%%%%%%%%%%%%%%%%%%%%%%%%%%%
\xchapter{Revisão Bibliográfica}{Este capítulo traz a fundamentação teórica relacionada aos assuntos abordados ao longo do trabalho e foi dividida em ...}
\label{cap:revisao}
%%%%%%%%%%%%%%%%%%%%%%%%%%%%%%%%%%%%%%%%%%%%%%%%%%%%%%

\section{Fundamentação Teórica} \textbf{ }
% Deve conter os conceitos obtidos na literatura e mencionadas no desenvolvimento

    Esta seção traz conceitos relacionados às .... dados geográficos e estatísticos e ...
    
    \subsection{Dados Geográficos e Estatísticos} \textbf{ } \newline

        Lorem ipsum dolor sit amet, consectetur adipiscing elit. Praesent ac tellus turpis. Donec vitae lorem odio. Sed luctus vestibulum libero eget pellentesque. Vivamus in mi turpis. Donec molestie feugiat sollicitudin. Nam et eleifend tortor. Cras lacinia, magna in tristique consequat, urna lorem placerat odio, id viverra mi nulla nec sem. Ut imperdiet, felis a hendrerit imperdiet, velit metus eleifend diam, ut convallis neque augue vitae leo. Morbi condimentum rhoncus faucibus. Fusce elit justo, semper lobortis blandit sed, lacinia et ligula. Pellentesque elit magna, vestibulum vitae adipiscing in, tincidunt eu erat. Nullam eu lectus vel nunc eleifend consectetur. Phasellus faucibus blandit nisi, eget fermentum erat tincidunt et. Aliquam ullamcorper varius nunc, nec eleifend dolor porttitor eu. Etiam eget nunc eu erat facilisis consequat sed accumsan urna. Nam vitae eros et justo iaculis gravida in sed nisl. Donec auctor gravida ipsum. Morbi sed ipsum tortor, quis ornare purus.
        
    \subsection{Painéis de Visualização de Informação - \textit{Dashboards}} \textbf{ } 

        %Exemplo de Citação. \citeonline() Deve estar no biblio.bib
        O desenvolvimento de \textit{dashboards} e a visualização de dados são tratados por \citeonline{few2006information} na obra ``Criação de \textit{dashboards} informativos: A comunicação visual eficaz dos dados''. Nela foram definidos aspectos que devem ser observados para a construção de painéis de visualização e estes foram descritos a seguir:

        %Exemplo de Listagem por Símbolos
        \begin{itemize}
            \item Informações organizadas para apoiar seu significado e uso;
            \item Consistência mantida para uma interpretação rápida e precisa;
            \item Visualização esteticamente agradável;
            \item \textit{Design} para uso como plataforma de lançamento;
            \item Avaliação da Usabilidade nos painéis criados.
        \end{itemize}

        \textbf{ } \\
        
        %Exemplo de Enumeração
        Exemplo de Enumeração
        \begin{enumerate}
            \item Exemplo de Item 1;
            \item Exemplo de Item 2.
        \end{enumerate}

        \clearpage %Quebra de Página
        
        Para \citeonline{few2006information}, painéis de visualização devem ser projetados de acordo com aspectos importantes do design visual e de usabilidade:

        %Exemplo de Citação Longa 
        \begin{quoting}[rightmargin=0cm,leftmargin=4cm]
        \begin{singlespace}
        {\footnotesize    
            Alguns aspectos importantes do \textit{design} visual do painel ainda precisam ser considerados. Um dos mais desafiadores é a necessidade de organizar muitos itens de informação, muitas vezes relacionados apenas pela necessidade do espectador de monitorá-los todos de uma maneira que não resulte em uma bagunça desordenada. Este arranjo deve apoiar as relações intrínsecas entre os vários itens e a maneira pela qual eles devem ser navegados e utilizados para apoiar a tarefa em questão. O design de um painel deve suportar seu uso de forma otimizada e transparente. O todo também deve ser agradável de se ver, ou será ignorado 
            (\citeonline{few2006information}, p. 138).
        }
        \end{singlespace}
        \end{quoting}
        
        \citeonline{few2006information} afirma ainda que a usabilidade é um aspecto a ser observado, e avaliações de usabilidade na área de conhecimento da \acf{IHC}, tem em suas origens as propostas e diretrizes de \citeonline{nielsen1990heuristic}. 


\section{Trabalhos Relacionados} \label{sec:trabrelac} \textbf{ }

% trazer exemplos de outros trabalhos que fazem algo na linha do seu trabalho.
    Foram reunidas algumas pesquisas correlatas ao problema de pesquisa que se pretende investigar, particularmente estudos para análises visuais de ...

    \textbf{ } 

    \citeonline{santos2018plataforma} propõe uma plataforma distribuída de mineração de dados para \textit{Big Data} e realiza um estudo de caso com notas fiscais de consumidores, disponibilizados pela Secretaria de Tributação do Rio Grande do Norte. Os trabalhos se aproximam em alguns aspectos como a manipulação, processamento e extração de conhecimento a partir de grande volume de dados